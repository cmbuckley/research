% chapter5.tex (Evaluation)

\chapter{Conclusion}
\label{ch:conc}

This thesis is concerned with network coding, an emerging field of information theory which allows us to obtain the optimum flow of information through a network. It has a number of real-world applications which could reduce network delays, improve reliability or otherwise increase throughput. Such a great number of technological advancements rely on the transmission of data across such a network --- radio and television, telecommunications and the Internet to name but a few --- that understanding any possible improvements that can be made to this transmission is of great interest to the professionals who monitor them. With increasing focus placed on wireless communication instead of wireline systems, the detection and correction of data corruption becomes ever-more important, as does the compression and security of the transmission. Network coding has a number of diverse applications to which it can be applied.

In attempting to understand problems in network flow, a mathematical representation of the problem is used. In Riis' guessing game, the coding functions used in the network are represented by the guessing strategies adopted by the players. There is a one-to-one correlation between the solutions of an information network flow problem and the optimal guessing strategies adopted by the players on the corresponding graph.

The main intention of the thesis was to combine the techniques of fractional routing with the concept of guessing numbers to further the study of guessing numbers and their use with network information flow problems. In attempting this, it was also hoped to be possible to answer some of the open questions relating to network coding. The evaluation shows that these objectives have been met, but that the method only succeeds in a limited number of cases. However, further work could go some way into increasing the number of problems where fractional routing provides the optimum solution.

The new guessing game is an intersection of two major areas of mathematics, and shows what can be achieved with a relatively simple example. Regarding the specific criteria for success, it is fair to say that grandiose objectives such as a ``new method'' were perhaps overly ambitious, and to suggest that this objective had been entirely met would be over-stating the achievements: the work produced is only a minor modification of work introduced in the literature. The formulation of completely new methods is left to the experts; the work in this thesis simply aims to build on those methods already suggested. Beyond this, however, it is fair to conclude that the objectives were both well-proposed and satisfactorily met.

A larger range of different examples would be required to show the general usefulness of the new guessing game to the network coding. The field of network coding is still rather new, and more proof is needed of its viability before it is widely adopted as an alternative to traditional routing.

\section{Open Questions}

% Adjust the spacing of the items to fit on the page (presentational markup)
\begin{itemize}\addtolength{\itemsep}{-0.2\baselineskip}
	\item{The routing capacity of a network is always rational. Is this true for a guessing number?}
 	\item{If not, what restrictions must be placed on the guessing number to force rationality?}
 	\item{If the guessing number $k$ of a given graph is independent of the alphabet size $s$, is $k$ necessarily an integer?}
\end{itemize}