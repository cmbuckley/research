% chapter5.tex (Evaluation)

\chapter{Evaluation}
\label{ch:eval}

This chapter provides an evaluation of the new guessing game based on network routing capacity. It looks at the objectives mentioned in the introduction of the thesis, and attempts to critically evaluate how successful the thesis was in achieving those objectives. It also suggests possible further work, as a number of ideas for further research have become apparent during the course of developing the methods used here.

\section{Evaluation of the Criteria for Success}

In \autoref{sect:objective} the following criteria for success were outlined:

\begin{itemize}
	\item{study and description of existing methods within network coding and network routing capacity;}
 	\item{development of a new method for solving network flow problems based on existing techniques;}
 	\item{abstraction and formalisation of a modified guessing game based on the new method;}
 	\item{study and evaluation of cases in which the new game is more successful than the existing game.}
\end{itemize}

These criteria are evaluated below to determine the overall success of the thesis.

\subsection{Study and Description of Existing Methods}

\autoref{ch:litrev} contains the literature review, which investigates existing work and methods currently in use. It follows on from the historic introduction provided in \autoref{ch:intro}, which provides a basic chronology of the events leading up to the beginning of this thesis. The chapter gives an in-depth explanation of the traditional routing methods in use today, and follows on to give simple examples where network coding has been shown to be superior. Guessing numbers and routing capacity are also defined here, along with a number of definitions which are used throughout the thesis. A number of other relevant topics which will be used in later sections are given detailed explanations, together with simple examples to further illustrate their meaning.

Each major technique for dealing with network flow problems has been either superseded or generalised to tackle a larger set of problems, and though specific approaches and notations may have changed and improved, the overall goal remains the same: to model in theoretical terms particular problems which occur in the real world, to understand them and to investigate how they may be solved. Work that brings together two or more techniques within a field --- like combining information theory with network routing \cite{ahls2000}, or indeed brings together work from two or more fields --- like network coding and graph theory \cite{riis2005util, riis2005info}, may be seen as an advancement towards some unified theory for understanding network flow. The literature survey has been satisfactory in determining the most useful features of existing work to be combined into the new game.

\subsection{Development of a New Method}

To this end, this thesis has brought together two recent techniques in network theory, namely Riis' largely uncharted field of guessing numbers and Cannons \textit{et al.}'s study into routing capacity. \autoref{ch:prob} arrives at the conclusion that these are the best areas to consider, from discussing a number of possible approaches to arriving at a suitable amalgamation to be formalised and studied.

\autoref{sect:ideas} lists some of the initial areas of research before moving on to study routing capacity. Throughout the research, a great number of ideas were discussed, before being disregarded due to their difficulty. It proved to be rather frustrating that so many questions were raised that simply could not be answered sufficiently, but it was necessary to discard these ideas in favour of more fruitful areas of research.

The main result of this development was to arrive at the combination of techniques mentioned above, extending and modifying the guessing game played on graph representations of network flow problems, to include the routing capacity of those networks. The game introduced is not \emph{new}, however, and is merely an extension of the game existing in the literature.

What the game does do is to include the concept of fractional routing for multiple-unicast networks into the existing guessing game: the original guessing game was concerned with using a guessing strategy over an alphabet $A$. The notion of fractional routing uses block coding where strings of letters $\{ a_1, a_2, \dots, a_k \}$ are used, which is mathematically equivalent to using an underlying alphabet of $A^k$.

\newpage

\subsection{Abstraction \& Formalisation of a Modified Guessing Game}

Once the extension was developed, the ideas proposed were formalised to produce the co-operative guessing game $\mathrm{RoutingGuessingGame}(G, s)$ played on a graph $G$ where the messages are selected from an alphabet of size $s$. This essentially incorporates the notion of fractional routing into the existing game. Particular problems were identified and examples were given in \autoref{ch:sol}. Particularly, the main result has been to show the effectiveness of the new game with odd cycles $C_{2k + 1}$, $k \in \NN$.

To further illustrate the place of the new game in relation to the current literature, a theorem is stated linking the fractional guessing game to fractional routing in the same way that the ordinary guessing game was related to network coding. The connection between the fractional guessing number and the fractional routing solution of the related circuit information flow problem helps to formalise the guessing game.

A general objective mentioned in the introduction was to go some way to reducing the number of open questions in the field of network coding and guessing numbers. When guessing numbers were first defined, they helped to answer some questions, but also introduced questions into the field. One such question was whether the guessing number of an undirected graph is always an integer \cite{riis2005util}. In \autoref{sect:pentagon-proof}, it is proven that the undirected pentagon $C_5$ has a non-integer guessing number. An unexpected result was that the guessing number also depended on the alphabet size $s$. This work is then generalised to odd cycles of the form $C_{2k + 1}$, for $k \in \NN$. It is these odd cycles where the newly introduced method has an advantage over the existing use of scalar linear network coding.

Although the thesis does indeed prove and give examples to satisfactorily answer this question, other questions arise, mostly regarding using this method in conjunction with existing techniques. For instance, it is not known whether a network exists for which both linear network coding and fractional routing are insufficient, yet some other method succeeds.

\subsection{Evaluation of the New Method against the Old Method}

The main objective mentioned in \autoref{sect:objective} was to give a new method of solving network flow problems. The result is that using a fractional routing approach, a class of networks has been found where this new approach improves on the success of linear network coding. Specifically, as mentioned in \autoref{pentagon}, there exists a simple graph (the $5$-cycle) where a guessing strategy using fractional routing (and two dice) fares better than any (fractional) guessing strategy using one die. Whilst this method does improve on both the existing guessing strategy, it does so in a very small class of networks. The idea of using a fractional guessing strategy would need to be continued to see if it succeeds on other classes of networks where other guessing strategies do not. It would be interesting to see if there is any case where it could out-perform a network coding approach using non-linear functions.

The correspondence between the guessing game defined here and the game previously defined in \cite{riis2005util} is not precise: messages in routing problems are generally larger than those in network coding. Consequently, it is difficult to provide a comparison between the games --- sometimes a coding strategy fares better, other times a fractional routing strategy proves more successful. However, this does mean that we have extended the existing approach for studying this class of information flow problems.

Whilst the idea of combining the techniques of guessing games and fractional routing may make sense in terms of data transmission, it does not do so in terms of the games outlined in the previous chapters. However, it may prove to have some validity if a restriction is placed on the network coding strategy, such as only allowing linear functions to be used.

\newpage

\section{Extensions}

One interesting extension has already been mentioned above, namely the combination of linear network coding with routing capacity to produce more advanced strategies for the guessing games. However, study in this field seems to be moving toward the use of non-linear coding functions to provide the best results in the case of larger networks. By comparison, the work in this thesis is on relatively simple networks, and may not translate to these more complex situations.

It may also be worth considering further study into the public-channel version of the guessing game, and extending that to include routing capacity in the same way that the ordinary game was altered. It is likely that this would back up the findings from the odd cycle: for each message to be uniquely determined on an odd cycle, a smaller public channel would be required for a fractional routing approach than for the existing approach.

Different representations of the same problem can yield different solutions, in some cases more successful than previous, known results. The examples given in \autoref{ch:sol} are cases where the fractional routing provides a solution which succeeds with higher probability than scalar linear solutions. Further extending these games would hopefully lead to a collection of approaches for tackling a great number of problems in network flow.